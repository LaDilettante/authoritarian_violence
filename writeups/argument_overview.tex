\section{The Argument}
\label{sec:argument}

This article argues that if the authoritarian legislature indeed helps institutionalize the regime-dissident interaction, then we should see decreased level of violence between the two sides in the presence of an authoritarian legislature. Taking a page from the rationalist war literature, I explain how political violence should only happen when the regime and the opposition misjudge each other's capacity and preference, or when the bargaining game does not have the option of side payments. Given transparent information, a commitment device, and regular channels to negotiate side payments, reaching an agreement is always preferable to violence due to the savings in the social cost of violence.

Armed with this theoretical argument, I empirically examine the level of violence in regime-dissident interaction using a novel event dataset across authoritarian regimes post Cold War. I construct the dataset by looking explicitly at countries that create a legislature during their authoritarian spells and matched these countries with those with the same conditions but do not create legislature. In addition to this large-n investigation, I zoom in at the causal mechanism with a survey experiment implemented in Vietnam, testing whether the recent inclusion of businessmen as delegates in the National Assembly prompts business to participate in the legislature. This survey experiment achieves two goals. First, it allows us to have a cleaner causal inference compared to large-n observational data, especially since violence is potentially too complex a process to model completely. Second, it investigates the immediate behavioral response of business to co-optation attempts, which serves as the logical pre-requisite to any secondary material outcomes that relies on the regime and the dissident interacting with each other. I find that the presence of a legislature does reduce violence across regimes, albeit with a modest substantive effect size since the model's flexibility lets most of the effects be captured in country and country-year intercepts. On the other hand, the survey experiment conclusively shows that the legislature successfully works as a co-optation device, prompting more willingness to participate in the legislature from business. These two findings together complete the missing links in our understanding of authoritarian institutions. They ascertain that the legislature does foster positive regime-opposition interaction, a necessary condition for regime durability, increased investment, and economic growth.