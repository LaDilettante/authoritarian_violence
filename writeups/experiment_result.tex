\subsection{Result}

Given the experimental design of our survey experiment, estimating the treatment effect is straight forward with a simple logistic regression as reported in \autoref*{tab:survey_result}.

The treatment is highly significant ($p < 0.01$) with a coefficient of 0.264. This means that, when prompted to think about the inclusion of businessmen in the National Assembly, the respondent is 1.3 time more likely to think that participating in economic policy making via the National Assembly is effective.

The size of this treatment effect is substantively significant given that the presence of businessmen in Vietnam's National Assembly is still limited. Furthermore, policy making in Vietnam is still dominated by the Executive branch, with the National Assembly mainly deliberating scripted agenda prepared by the Standing Committee \citep{Montesano2005}. Despite these limitations on the participation of business in policy making through the National Assembly, we see that business are still willing to participate in the legislature. Therefore, the treatment effect in this survey experiment is likely to be even larger in more open authoritarian legislatures.

Given this result, we are more confident that the positive effect of legislature on reducing regime-opposition violence is genuine. Indeed, the willingness of the political outsiders to partipate and communicate when given the chance is the necessary ingredient for the legislature to function as an informational and commitment device.

\begin{table}[H] 
  \centering 
  \caption{Treatment effect in Survey experiment} 
  \label{tab:survey_result} 
\begin{tabular}{@{\extracolsep{5pt}}lc} 
\\[-1.8ex]\hline 
\hline \\[-1.8ex] 
 & \multicolumn{1}{c}{\textit{Dependent variable:}} \\ 
\cline{2-2} 
\\[-1.8ex] & Contacting the National Assembly is effective \\ 
\hline \\[-1.8ex] 
 Treatment prompt & 0.264$^{***}$ \\ 
  & (0.073) \\ 
  & \\ 
 Constant & 0.929$^{***}$ \\ 
  & (0.049) \\ 
  & \\ 
\hline \\[-1.8ex] 
Observations & 3,951 \\ 
Log Likelihood & $-$2,253.137 \\ 
Akaike Inf. Crit. & 4,510.274 \\ 
\hline 
\hline \\[-1.8ex] 
\textit{Note:}  & \multicolumn{1}{r}{$^{*}$p$<$0.1; $^{**}$p$<$0.05; $^{***}$p$<$0.01} \\ 
\end{tabular} 
\end{table} 


