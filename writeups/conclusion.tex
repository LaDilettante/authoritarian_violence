\section{Conclusion}
\label{sec:conclusion}

In this article, I have shown that authoritarian legislature does prompt the opposition to engage with the regime, resulting in regularized and transparent interactions that reduce the level of violence. These two findings contribute to the literature on authoritarian institutions by empirically ascertain two behavioral patterns in regime-opposition interaction that undergird all secondary material outcomes in the extant literature. Indeed, regime survival, increased investment, and economic performance are outcomes far down the causal chain from authoritarian institutions, the behavioral outcomes investigated in this paper most directly follow from institutional change and thus are the best evidence for its effect.

On the other hand, using event data with a high level of granularity, the cross-national hierarchical model also reveals that much of variation in violence across countries and times has not been satisfactorily explained. This finding calls for greater attention to the variation across authoritarian regimes, where institutions with the same name may serve different functions.