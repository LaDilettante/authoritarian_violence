\section{Introduction}

Once disregarded as mere ``window-dressing'', authoritarian institutions have garnered the attention of scholars as an important tool for power sharing, which, in turn, leads to various substantive effect, including regime survival, increased investment, better economic performance.

While our understanding of authoritarian institution has branched to numerous substantive topics, its causal root deserves more careful studies. Given its common definition as the ``rule of the game'', institutions are fundamentally a set of constraints of human interactions, and thus its effect must be most evident in the interaction between the strategic actors itself.  When scholars study the effect of institutions on various substantive areas, inevitably the causal chain must pass through the strategic actions of the actors, mostly through the logic of credible and regular spoil sharing (cite Przworski, Svolik). This causal step is theoretically crucial to all of our findings regarding authoritarian institutions – the goal of this article is, therefore, to empirically examine it.

Beyond its theoretical contribution, a study of violence in partially liberalized dictatorship has immense real-world importance. Indeed, violence during political liberalization is the greatest concern of many would-be democrats living under authoritarian regimes. For example, Chinese citizens show overwhelming enthusiasm when asked about democratic features such as popular accountability, separation of power, and political liberalism, yet remain consistently cautious of instability caused by transition (Chu 2008, 309). Understanding whether and why countries experience increased violence during liberalization is crucial for the prospect of democracy.

\section{Argument and Structure}

This article argues that if the authoritarian legislature is indeed a credible commitment and informational device, then we should see decreased level of violence between the regime and the dissdent in the presence of an authoritarian legislature. Indeed, taking a page from the rationalist war literature, I explains how political violence should only happen when the regime and the opposition misjudge one another's capacity and preference. Given transparent information and a commitment device, regardless of the capacity and preference of both sides, negotiating to reach an agreement is always preferable to violence due to the savings in the social cost of violence.

Armed with this theoretical argument, I empirically examine the level of violence in regime-dissident interaction using a novel event dataset across authoritarian regime post Cold War. I constructed the data by looking explicitly at countries that create their legislature during their authoritarian spells and matched these countries with those of the same conditions but did not create legislature. In addition to this large-n investigation, I zoom in at the causal mechanism with a survey experiment implemented in Vietnam, a hard case for the theory of legislature on regime-dissident interaction. These studies show that blah blah blah.

The article will proceed as follows: