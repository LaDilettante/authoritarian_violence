\section{Introduction}
\label{sec:introduction}

Once disregarded as mere ``window-dressing,'' authoritarian institutions have garnered the attention of scholars as an important tool for power sharing, which, in turn, leads to various substantive effects, including regime survival, increased investment, and better economic performance \citep{Gandhi2008}.

While our understanding of authoritarian institution has branched to numerous substantive topics, its causal root deserves more careful studies. Given its common definition as the ``rule of the game'', institutions are fundamentally a set of constraints on human interactions, and thus its effect must be most evident in the interaction between strategic actors itself.  When scholars study the effect of institutions on various substantive areas, inevitably the causal chain must pass through the strategic actions of the actors, mainly with the logic of credible and regular spoil sharing (Svolik, Przeworski). This causal step is theoretically crucial to all of our findings regarding authoritarian institutions. The goal of this article is, therefore, to empirically examine it.

Beyond its theoretical contribution, a study of violence in partially liberalized dictatorship has immense real-world importance. Indeed, violence during periods of political liberalization is the greatest concern of many would-be democrats living under authoritarian regimes. For example, Chinese citizens show overwhelming enthusiasm when asked about democratic features such as popular accountability, separation of power, and political liberalism, yet remain consistently cautious of instability caused by transition (Chu 2008, 309). Therefore, understanding the pattern of violence during liberalization is crucial for the prospect of democracy.

The article will proceed as follows. \Cref{sec:argument} presents an overview of the theoretical argument and empirical analyses. \Cref{sec:theory} posits two theoretical explanations for the effect of legislature on violence. \Cref{sec:empirics_largeN} uses a Bayesian hierarchical model to analyze political violence in authoritarian regimes post-Cold War, showing that the legislature has a modest positive effect. \Cref{sec:empirics_survey} uses a survey experiment to demonstrate that co-optation strategy does prompt Vietnamese business to interact with the regime, a pre-requisite for any material outcome such as violence to change. \Cref{sec:conclusion} concludes.