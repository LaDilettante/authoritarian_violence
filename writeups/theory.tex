\section{Theory: Authoritarian legislature and its effect on political violence}
\label{sec:theory}

\subsection{Legislature reducing violence}

To consider whether violent means is chosen to reach their goal, we need a theoretical framework to understand how the opposition and the regime engage in the bargaining game. A possible starting point is the bargaining models developed for democratic legislatures, where bargaining outcomes are driven by various institutional rules (positive and negative rule, sequencing of these powers, first-mover advantage, etc.) (Baron and Ferejohn 1989; Cox 2006). However, this type model is not suited to analyze the bargaining under an authoritarian regime, where the institutional rules themselves are under constant threat of being revised by the final arbiter that is violence. In other words, the end game of legislative bargaining now has an extra node: whether either side will violently suppress the other’s demand. Given this context, the decision of the autocrat and the opposition is better characterized as a bargaining game in anarchy, very much like the bargaining between two sovereign states with the option of going to war (Fearon 1995).

To formally model the interaction, let the negotiation between the opposition and the regime be about the share $x$, $(0 < x < 1)$ of a good (normalized as 1) given to the opposition.\footnote{For concreteness we can think of this as the share of the economy’s rent or capital stock}. Both sides have the option 1) to bargain and decide on x, or 2) to engage in violent protest / repression. Violence is costly compared to peaceful negotiation---we call this difference in relative cost $c_A$ and $c_O$ for the autocrat and the opposition, respectively. If violence happens, whichever side emerges victorious decides the share x to his heart’s desire---i.e. the winning autocrat would give 0 to the opposition and the winning opposition would give 1 to itself.

Given the uncertain nature of collective action battles, the utility of engaging in violence for the two sides is best formulated as an expected utility that depends on the probability of the opposition winning (call this probability $p$). Thus, the expected utility of the opposition is $p – c_O$, while the expected utility of the ruling regime is $1 - p - c_A$. These two expected utilities sum to less than one ($1 - c_A - c_O < 1$), which raises the puzzle why two sides fail to engage in ex-ante bargaining that splits the difference, giving both sides a higher payoff than the expected utility of violence. Indeed, any amount of x between $(p - c_O, p + c_A)$, i.e. the bargaining range, is more desirable than violence to both sides.

Yet war happens between states and so does violence under authoritarian regimes. The international relations literature suggests three factors that may lead to violence, all of which having a legislature can help alleviate.

\begin{enumerate}
\item Information asymmetry regarding $p$

Due to the incentive to bluff, both sides only have incomplete information about the capability of the other. This leads to differing estimate of $p$, leading to non-overlapping preferences. This information asymmetry can be alleviated by having an elected legislature that helps the autocrat identify the collective action capability of the opposition. During regular legislative sessions, the regime can observe the position of opposition legislators to gauge their unity strength. Furthermore, during elections, both sides’ mobilization effort provides information to the other about its collective action capability without actually resorting to violence. The real, competitive stake of these elections also means that both sides are exerting its best effort, reducing the risk of bluffing and deception. Therefore, the autocrat can trust the information it receives about the opposition’s capability $(p)$, and consequently has a better idea about the political constraint that it faces.

\item The bargaining range $(p - c_O, p + c_A)$ depends on the value of $c_O$ and $c_A$

Having an elected legislature can also increase $c_A$ and $c_O$, i.e. the cost of violence compared to peaceful negotiation. First, elections provide the opposition with a platform to influence policy making without resorting to violence (Rigger 1999, 14). Despite inevitable electoral manipulations, voters still know who the candidates are and can vote in support of their message. While these votes will not dethrone the regime, they serve as an informal referendum that alerts the regime about the popularity of its policies. In this way, the opposition can influence the agenda without winning the election itself. Indeed, if participation in the electoral game proves to be an attractive option to the opposition, radicalizing becomes relatively more costly as a means of gaining influence.

Second, in Dahlian terms, elections also tempt the ruling regime to stay within the institutionalized game by raising the cost of repression (Dahl 1971, 15). Indeed, elections “open up avenues of collective protest” by creating convergent social expectations that are both large in scope (affecting all citizens) and concentrated in time frame (Schedler 2009). Furthermore, by participating in elections, the opposition builds up its organizational strength, which can be used to mobilizing votes as well as organizing protest (Thompson and Kuntz 2004).

In addition, upholding elections and institutional bargaining is an attractive option for regimes to build their legitimacy and assuage grievances about past infractions (Lindberg 2009, 339). By adapting its policies based on election results, the regime can also bring its position closer to the popular will in incremental steps, each with acceptable cost. Indeed, it is not contradictory to say that an authoritarian regime may protect voters’ rights or respond to their demands as a strategy for the regime to stay in power, because there are often hard-liners and reformists within the regime. If the reformists’ strategy of electoral engagement proves to be a successful alternative to maintain power, more members of the regime will be tempted to become liberal-minded.

\item Unit indivisibility: the contested good may not be divided continuously

A final possibility that leads to violence is that the opposition and the ruling regime may be fighting over issues that cannot be continuously divided. For example, were the contested issue about whether religious law should be adopted, it may not be possible to split the issue space to arrive at a compromise within the (p - cO, p + cA) range. An elected legislature again ameliorates this problem. By providing an arena in which both sides can engage in multiple negotiations along many policy issues, it provides more opportunities for side payments. In contrast, were the opposition shut out of the policy making entirely, the only way they could gain influence is to arouse the populace to violent protest. The best way to do so is not via a nuanced discussion of multiple policy points but an intense focus on a highly contentious issue, further exacerbating the problem of unit indivisibility. Therefore, we should expect elected legislature to provide better opportunities for negotiation.

In sum, the overarching of over these causal mechanisms is that the elected legislature provides an arena for the ruling and the opposition to interact, gain information, and adjust strategies. In game theoretic sense, by lengthening the horizon and improve the regularity of stakes, repeated interaction encourage players to think about the future and deter the temptation to radicalize. Overall, these effects encourage incremental changes over radical tactic and favor cooperation over defection (Axelrod 1984). This leads to our hypothesis that:

\begin{quote}
H1: Authoritarian regimes that create a legislature will experience less violence in their interaction with the dissidents.
\end{quote}
\end{enumerate}

\subsection{Legislature increasing violence}
While the prevalent theory of legislature as an informational and commitment device favors hypothesis H1, it is important to note that there are reasons to susepct the opposite. Throughout the previous model, it is implied that the actors’ utility functions remain static with the only parameters being the cost of violence and the share of the issue space. Yet this assumption may miss the dynamism of the opposition’s democratic expectation, which becomes increasingly demanding as the regime starts liberalizing.

Speaking about this issue, Ted Gurr hypothesizes, ``The potential for collective violence varies strongly with the intensity and the scope of relative deprivation among members of a collectivity.'' Relative deprivation, in turn, is defined as the perceived discrepancy between value expectation (what people believe they are entitled to) and value capability (what the people believe they are able to attain) (1970, 24). Seen under this light, repeated elections may increase the potential for political violence by raising the democratic expectation, which can lead to violent protest if unfulfilled. The effect of elections is especially potent because it affects all citizens, thus raising not only the intensity but also the scope of relative deprivation.

Indeed, unfulfilled political expectation is a common case of relative deprivation causing violence. Historical examples are numerous, spanning various time periods as well as scopes of conflict: from the Puritan Revolution of 1640-60, which was sparked by attempts of the Stuart kings to reinstate royal absolutism, to the protest in twentieth-century colonies, which often followed the imposition of restrictions after a period of political liberalization (1970, 115). This pattern makes sense because, when people have tasted freedom, not only do they want more rights but they also consider rights more important. In other words, the opposition may not simply optimize its share of the issue space but also develops a minimum level of acceptable concession. If opposition’s expectation rises too quickly, the ruling regime may not have enough time to appropriate adjust their position especially if the hard-liners are also trying to retrench given signs of a political spiral. For this reason, it is plausible that electoral authoritarianism leads to more violent interaction between the ruling regime and the opposition, leading to the second hypothesis:

\begin{quote}
H2: Authoritarian regimes that create a legislature will experience more violence in their interaction with the dissidents.
\end{quote}