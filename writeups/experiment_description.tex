\subsection{Design}

From the large-n analysis, we find suggestive evidence of an authoritarian legislature having a moderating effect on political violence. However, while the matching design alleviates our concern about functional misspecification and incorrect counterfactual, the threat of selection on unobservable remains. Indeed, large variation in the country and country-year intercepts suggests that our model of violence is not yet precise.

Therefore, I follow up the large-n analysis with a survey experiment that aims to affirm a causal prerequisite for my argument. The posited theory about the legislature functioning as an informational and commitment device hinges on the assumption that having legislature actually prompts the co-opted elites to participate in the forum. Otherwise, if the opposition does not consider the legislature to be an effective arena for policy change, then neither can the legislative bargaining provide an attractive alternative to violence nor can the regime gain information about the opposition's capability.

The survey experiment is as follows. Conceptually, the treatment is a country ``having an elected legislature,'' which cannot be randomized. However, in our survey design we can prompt the respondents to think about the legislature as a bargaining arena. My survey question is included in Vietnam's annual business survey PCI (Provincial Competitiveness Index) as follows:

\begin{quote}
(Control prompt) We want to understand how business participate in the process of economic policy making

(Treatment prompt) We want to understand how business participate in the process of economic policy making, \textit{especially since the National Assembly has recently included more delegates that are businessmen} (emphasis added)

Question: Approaching National Assembly delegates to express opinions about economic polices is effective or not?
\end{quote}

Vietnam is an appropriate case for the experiment because Vietnam's growing business sector is being co-opted into the National Assembly. Like China, Vietnam has liberalized its economy while resisting political liberalization. In its 1992 Constitution, drafted when economic reform was already started, the Vietnamese Communist Party (VCP) asserted its role as the vanguard party that commanded the leadership role of the country. At the same time, economic liberalization has created a new class of private entrepreneurs, whose growth becomes increasingly crucial to the country's prosperity and the regime's stability especially when the state-owned sector continues to sluggishly perform. 

Given this mismatch between political and economic power, Vietnam is an insightful case study of the co-optation theory with the new class of capitalists being the potential opposition. Indeed, after the 1992 Constitutional revision, the Vietnamese National Assembly (VNA) has included more diverse and assertive, amending law proposed by the executive, albeit without rejecting it. The representation of businessmen and women in the National Assembly has also increased. In 2007 only 1 out of 30 self-nominated candidates won seat in the VNA, while in 2011 4 out of 15 candidates won seats, including two of the country's best known capitalists. (cite http://blogs.reuters.com/global/2011/06/04/party-wins-big-in-vietnam/)

