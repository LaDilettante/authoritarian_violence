\documentclass[12pt]{article}

% This first part of the file is called the PREAMBLE. It includes
% customizations and command definitions. The preamble is everything
% between \documentclass and \begin{document}.

\usepackage[margin=1in]{geometry}  % set the margins to 1in on all sides
\usepackage{graphicx}              % to include figures
\usepackage{amsmath}               % great math stuff
\usepackage{amsfonts}              % for blackboard bold, etc
\usepackage{amsthm}                % better theorem environments


% various theorems, numbered by section

\newtheorem{thm}{Theorem}[section]
\newtheorem{lem}[thm]{Lemma}
\newtheorem{prop}[thm]{Proposition}
\newtheorem{cor}[thm]{Corollary}
\newtheorem{conj}[thm]{Conjecture}

\DeclareMathOperator{\id}{id}

\newcommand{\bd}[1]{\mathbf{#1}}  % for bolding symbols
\newcommand{\RR}{\mathbb{R}}      % for Real numbers
\newcommand{\ZZ}{\mathbb{Z}}      % for Integers
\newcommand{\col}[1]{\left[\begin{matrix} #1 \end{matrix} \right]}
\newcommand{\comb}[2]{\binom{#1^2 + #2^2}{#1+#2}}


\begin{document}


\nocite{*}

\title{A Sample Mathematics Paper}

\author{Anh Le\thanks{Grant support listed here.} \\ 
Department of Political Science \\
Duke University \\
Durham, NC 27708 USA}

\maketitle

\begin{abstract}
  Mathematical model of my authoritarian violence paper
\end{abstract}

\section{Multilevel model}

\begin{align}
&\text{goldstein}_i &&\sim N(\alpha^{countryyear}_{j[i]}, \sigma^2_{goldstein}) \\
&\alpha_j^{countryyear} &&\sim N(\beta_{k[j]}^{country} + \gamma \cdot \text{liec}_j + \Gamma X_j, \sigma_{\alpha}^2) \\
&\beta_k^{country} &&\sim N(\delta_{ethnic} \cdot \text{ethnic}_k, \sigma_\beta^2) 
\end{align}


\section{Multilevel model with instrumental variable}

Unit indexed by $i$, group indexed by $j$. Model with treatment $T$ and IV $z$ both at the group level. There are also covariates at the group level, and none at the individual level.

\begin{align}
y_i &\sim N(\alpha_{j[i]}, \sigma_y^2) \\
\col{\alpha_j \\ T_j} &\sim 
N \left(
    \col{\gamma_0 + \gamma_1 T_j + \gamma_2 x_j\\\mu_0 + \mu_1 z_j} ,
    \left[
    \begin{matrix}
      \sigma^2_\alpha & \rho \sigma_\alpha \sigma_T \\
      \rho \sigma_\alpha \sigma_T & \sigma^2_T
    \end{matrix}
    \right]
\right)
\end{align}

In our paper,

\begin{align}
T_j &= \text{legislature in country-year}_j \\
z_j &= \text{inherited parties in country-year}_j \\
x_j &= \text{other covariates in country-year}_j
\end{align}

Notice that in the model, $T_j$ is not assumed to be uncorrelated with the error term of $\alpha_j$ (the non-diagonal covariance is non-zero), but $z_j$ is (implicitly since it's un-modeled). Writing it out in non-multilevel form:

\begin{align}
\alpha_j &= \gamma_0 + \gamma_1 T_j + \gamma_2 x_j + \epsilon_\alpha = N(\gamma_0 + \gamma_1 T_j + \gamma_2 x_j, \sigma_\alpha)\\
T_j &= \mu_0 + \mu_1 + \epsilon_T = N(\mu_0 + \mu_1, \sigma_T)
\end{align}

In the multilevel model, we allow $T_j$ to be correlated with $\epsilon_\alpha$ via $\rho \sigma_\alpha \sigma_T$


\end{document}
