\subsection{The model}

Given that my dependent variable of interest (\textit{violence}) is at the event level, while my ``treatment'' (\textit{legislature}) is at the country year level, a Bayesian hierarchical model as the most appropriate choice.\footnote{The choice of Bayesian hierarchical model is largely practical: with 15,037 events, popular R packages to fit frequentist hierarchical model do not converge due to a high chance of numerical issues.}  Each event belongs to a country-year, and each country-year belongs to a country. The intercept at each level varies according to the group the unit belongs to. This model choice is superior to the alternative option, which is to aggregate the event-level data up to the country-year level and run a fixed effect model. Doing so will 1) ignore the variation at the event level, resulting in standard errors that are too small; 2) leave the aggregation mechanisms up to the researcher with no clear theoretical reasoning.\footnote{Should we measure the level of violence within a country year by averaging across events? Why not the median? Or count the number of positive / negative events? The choice is not likely to be theoretically driven and open to abuse.}

All parameters have conjugate un-informative prior so that the result is driven entirely by the data instead of the prior.\footnote{More specifically, the prior for coefficient estimates is $Normal(\text{mean}=0, \text{precision}=.0001)$. The prior for the variance parameters is $Inverse-Gamma(1, 1)$} The model is thus:

\begin{alignat*}{2}
&{event}_i &&\sim N(\alpha^{countryyear}_{j[i]} + A \cdot \text{sector}_i, \sigma^2_{event}) \\
&\alpha_j^{countryyear} &&\sim N(\beta_{k[j]}^{country} + \gamma \cdot \text{legis}_j + \Gamma X_j, \sigma_{\alpha}^2) \\
&\beta_k^{country} &&\sim N(\delta_0 + \delta_{ethnic} \cdot \text{ethnic}_k + D \cdot \text{region}, \sigma_\beta^2) 
\end{alignat*}

At the event level:

\begin{conditions*}
\alpha^{countryyear}_{j[i]}    &  intercept of $countryyear_j$ to which $event_i$ belongs\\
\text{sector}_i     &  matrix of dummies for the dissident sector of $event_i$  \\


\end{conditions*}

At the country year level:
\begin{conditions*}
\beta_{k[j]}^{country} &  intercept of $country_k$ to which $countryyear_j$ belongs. In traditional panel data model, this is the country fixed effect. \\
\text{legis}_j & dummy variable for legislature. This is our parameter of interest. \\
X_j & matrix of other control variables, including log(GDP), log(GDP per capita), military expenditure (as \% of GDP), oil and mineral export (as \% of GDP), regime duration (years), and authoritarian typology (military, personal, and party as coded by Geddes et. al)
\end{conditions*}

At the country level:
\begin{conditions*}
\text{ethnic} & level of ethnic fractionalization as defined by Fearon \\
\text{region} & matrix of regional dummies
\end{conditions*}

I fit this model with a Gibbs sampler written in JAGS, running 3 chains, each has 11,000 iterations with 1,000 burn-ins. The mixing is excellent and the result is consistent across 3 chains. More model diagnostics details can be seen in \autopageref{sec:modeldiagnostic}